\usepackage[T1]{fontenc}
\usepackage{charter}
\usepackage[expert]{mathdesign}
\usepackage{amsmath,amssymb,amsthm,amsfonts}
\usepackage{mathtools}
\usepackage{minitoc-hyper}
\usepackage{graphicx,wrapfig}
\graphicspath{{./figures/}}
\usepackage[a4paper,width=150mm,top=25mm,bottom=25mm,bindingoffset=6mm]{geometry}
\usepackage{emptypage}
\usepackage[cal=boondox]{mathalfa}
\usepackage{wasysym,textcomp,array,arcs,polynom,cancel,enumerate,dcolumn} 
\usepackage{float}
\usepackage{caption}
\usepackage{subcaption}
\usepackage{mathrsfs}
% \usepackage[lite]{mtpro2} % for centered widetilde
% \usepackage{showlabels}
\usepackage{makeidx}
\makeindex 
\usepackage{afterpage}
\usepackage[all, cmtip]{xy} 
\usepackage[Lenny]{fncychap} %Options: Sonny, Lenny, Glenn, Conny, Rejne, Bjarne, Bjornstrup
\usepackage{fancyhdr}
\pagestyle{fancy}
\fancyhf{}
\fancyhead[LE,RO]{\thepage}
\fancyhead[RE]{\leftmark}
\fancyhead[LO]{\rightmark}
%\fancyfoot[CE,CO]{\leftmark}
%\fancyfoot[LE,RO]{\thepage}

\allowdisplaybreaks
% \usepackage[pagewise]{lineno}\linenumbers
% \usepackage{ebgaramond} 
\usepackage[usenames,dvipsnames]{xcolor}
%\usepackage{sectsty}

\definecolor{aureolin}{HTML}{009EFF}
\definecolor{tealGreen}{HTML}{008731}
\usepackage[linktoc=all,colorlinks,linkcolor=teal,citecolor=aureolin]{hyperref}
\usepackage[nameinlink,noabbrev]{cleveref}
\crefname{defn}{definition}{definitions}
\Crefname{defn}{Definition}{Definitions}
\usepackage[symbols,nogroupskip,stylemods,postpunc=dot, automake]{glossaries-extra}
\usepackage{tikz-cd}
\usepackage{cases} % for equation numbering in cases
\usepackage{longtable}
\makeatletter
% \usepackage{refcheck}

% \newcommand\frontmatter{%
% 	\cleardoublepage
% 	%\@mainmatterfalse
% 	\pagenumbering{roman}}

% \newcommand\mainmatter{%
% 	\cleardoublepage
% 	% \@mainmattertrue
% 	\pagenumbering{arabic}}

% \newcommand\backmatter{%
% 	\if@openright
% 	\cleardoublepage
% 	\else
% 	\clearpage
% 	\fi
% 	% \@mainmatterfalse
% }

% \makeatother


% NewCommands

\newcommand{\Rbb}{\ensuremath{\mathbb{R}}} 
\newcommand{\rbb}{\ensuremath{\mathbb{R}}} 
\newcommand{\cbb}{\ensuremath{\mathbb{C}}} 
\newcommand{\zbb}{\ensuremath{\mathbb{Z}}} 
\newcommand{\nbb}{\ensuremath{\mathbb{N}}} 
\newcommand{\vectorsall}{\ensuremath{\mathbf{v}_1,\mathbf{v}_2,\ldots,\mathbf{v}_n}} 
\newcommand{\vectorsallw}{\ensuremath{\mathbf{w}_1,\mathbf{w}_2,\ldots,\mathbf{w}_n}} 
\newcommand{\pbf}{\ensuremath{\mathbf{p}}} 
\newcommand{\xbf}{\ensuremath{\mathbf{x}}} 
\newcommand{\ybf}{\ensuremath{\mathbf{y}}} 
\newcommand{\zbf}{\ensuremath{\mathbf{z}}} 
\newcommand{\ubf}{\ensuremath{\mathbf{u}}} 
\newcommand{\vbf}{\ensuremath{\mathbf{v}}}
\newcommand{\wbf}{\ensuremath{\mathbf{w}}}
\newcommand{\ebf}{\ensuremath{\mathbf{e}}}
\renewcommand{\bf}[1]{\ensuremath{\mathbf{#1}}}
\newcommand{\sbb}{\ensuremath{\mathbb{S}}}
\newcommand{\cali}[1]{\ensuremath{\mathcal{#1}}}
\newcommand{\scr}[1]{\ensuremath{\mathscr{#1}}}
\newcommand{\mscr}{\ensuremath{\mathscr{M}}}
\newcommand{\nscr}{\ensuremath{\mathscr{N}}}
\renewcommand{\sf}[1]{\ensuremath{\mathsf{#1}}}
\newcommand{\red}[1]{\textcolor{red}{#1}} 
\newcommand{\blue}[1]{\textcolor{blue}{#1}} 
\newcommand{\brown}[1]{\textcolor{brown}{#1}} 
\newcommand{\magenta}[1]{\textcolor{magenta}{#1}} 
\newcommand{\cyan}[1]{\textcolor{cyan}{#1}}
\newcommand{\bb}[1]{\ensuremath{\mathbb{#1}}} 
\newcommand{\delbydel}[2]{\ensuremath{\dfrac{\partial#1}{\partial#2}}} 
\newcommand{\crf}{\ensuremath{\mathrm{Cr}(f)}} 
\newcommand{\grad}{\ensuremath{\nabla}} 
\newcommand{\comp}{\mathbin{\mathchoice {\compcent\scriptstyle}{\compcent\scriptstyle} {\compcent\scriptscriptstyle}{\compcent\scriptscriptstyle}}} 
\newcommand{\compcent}[1]{\vcenter{\hbox{$#1\circ$}}} 
\newcommand{\dbb}{\ensuremath{\mathbb{D}}} 
\newcommand{\dist}{\ensuremath{\operatorname{dist}}} 
\newcommand{\dbyd}[2]{\ensuremath{\dfrac{d#1}{d#2}}} 
\newcommand{\isom}{\ensuremath{\cong}} 
%\newcommand{\cal}[1]{\ensuremath{\mathcal{#1}}} 
\newcommand{\acal}{\ensuremath{\cal{A}}}
\newcommand{\im}{\ensuremath{\operatorname{im}}}
\newcommand{\defeq}{\vcentcolon=}
\newcommand{\eqdef}{=\vcentcolon}
\newcommand{\define}{\overset{\mathrm{def}}{=\joinrel=}}
\newcommand{\tensor}{\ensuremath{\otimes}}
\newcommand{\directsum}{\ensuremath{\oplus}}
\newcommand{\homeo}{\ensuremath{\cong}}
\newcommand{\cutn}[1][N]{\ensuremath{\mathrm{Cu}(#1)}}
\newcommand{\sen}{\ensuremath{\mathrm{Se}(N)}}
\newcommand{\innerprod}[2]{\ensuremath{\left\langle #1,#2\right\rangle}}
\newcommand{\norm}[1]{\ensuremath{\left\|#1\right\|}}
\newcommand{\range}{\ensuremath{\mathrm{range}}}
\newcommand{\paran}[1]{\ensuremath{\left( #1 \right)}}
\newcommand{\curlybracket}[1]{\ensuremath{\left\{ #1 \right\}}}
\newcommand{\squarebracket}[1]{\ensuremath{\left[ #1 \right]}}
\newcommand{\abs}[1]{\ensuremath{\left|#1\right|}}
\newcommand{\aTransInverse}{\ensuremath{\paran{A^T}^{-1}}}
\newcommand{\sqrtATransAInverse}{\ensuremath{\paran{\sqrt{A^TA}}^{-1}}}
\newcommand{\trace}[1]{\ensuremath{\mathrm{tr}\left( #1 \right)}}
\newcommand{\cu}{\ensuremath{\mathrm{Cu}(p)}}
\newcommand{\se}[1][p]{\ensuremath{\mathrm{Se}(#1)}} 
\newcommand{\co}{\ensuremath{\mathrm{Co}(x_0,\delta)~}} 
\newcommand{\costar}{\ensuremath{\mathrm{Co}^\star(x_0,\delta)~}}
\newcommand{\Ball}{\ensuremath{\overline{B(x_0,\delta)}~}}
\newcommand{\scal}{\mathcal{S}}
\newcommand{\bcal}{\mathcal{B}}
\newcommand{\hess}{\mathrm{Hess}}
\newcommand{\R}{\mathbb{R}}
\newcommand{\C}{\mathbb{C}}
\newcommand{\CP}{\mathbb{CP}}
\newcommand{\sss}{\mathcal{S}}
\newcommand{\bgd}{\begin{displaymath}}
\newcommand{\edd}{\end{displaymath}}
\newcommand{\bgc}{\begin{center}}
\newcommand{\edc}{\end{center}}
\newcommand{\hf}{\hspace*{0.5cm}}
\newcommand{\hfb}{\hspace{1cm}}
\newcommand{\lan}{\left\langle}
\newcommand{\ran}{\right\rangle}
\newcommand{\upq}{U(p,q)}
\newcommand{\ep}{\varepsilon}
\renewcommand{\epsilon}{\varepsilon}
\DeclareMathOperator{\spn}{span}
\newcommand{\bigzero}{\mbox{\normalfont\Large\bfseries 0}}
\newcommand{\ubb}{\mathcal{u}}
\newcommand{\vbb}{\mathcal{v}}
\newcommand{\rs}{0.7ex}


% % -------------------New commands for todo notes-------------------------------
% \newcommand{\doubt}[1]{\todo[color= red!40]{#1}}
% \newcommand{\change}[1]{\todo[color= green!40]{#1}}
% \newcommand{\info}[1]{\todo[color= blue!40]{#1}}
% \newcommand{\improvement}[1]{\todo[color= magenta!50]{#1}}
% \newcommand{\thiswillnotshow}[1]{\todo[disable]{#1}}

\newcommand{\spmat}[1]{%
  \left(\begin{smallmatrix}#1\end{smallmatrix}\right)%
}

\DeclareMathAlphabet{\mathpzc}{OT1}{pzc}{m}{it} 
\definecolor{PropColor}{HTML}{0DC8F2}
\definecolor{CorColor}{HTML}{FFC300}
\definecolor{ProbColor}{HTML}{FF1212}

% Theorems 
\usepackage{thmtools}
\usepackage[framemethod=TikZ]{mdframed}
\mdfsetup{skipabove=1em,skipbelow=0em}
\theoremstyle{definition}
\declaretheoremstyle[
    headfont=\bfseries\sffamily\color{Orange!70!black}, bodyfont=\normalfont,
    mdframed={
        linewidth=2pt,
        rightline=false, topline=false, bottomline=false,
        linecolor=Orange, backgroundcolor=Orange!5,
    }
]{thmthmbox}


\declaretheoremstyle[
    headfont=\bfseries\sffamily\color{NavyBlue!70!black}, bodyfont=\normalfont,
    mdframed={
        linewidth=2pt,
        rightline=false, topline=false, bottomline=false,
        linecolor=NavyBlue, backgroundcolor=NavyBlue!5,
    }
]{thmdefnbox}

\declaretheoremstyle[
    headfont=\bfseries\sffamily\color{LimeGreen!70!black}, bodyfont=\normalfont,
    mdframed={
        linewidth=2pt,
        rightline=false, topline=false, bottomline=false,
        linecolor=LimeGreen, backgroundcolor=LimeGreen!5,
    }
]{thmlemmabox}

\declaretheoremstyle[
    headfont=\bfseries\sffamily\color{PropColor!70!black}, bodyfont=\normalfont,
    mdframed={
        linewidth=2pt,
        rightline=false, topline=false, bottomline=false,
        linecolor=PropColor, backgroundcolor=PropColor!5,
    }
]{thmpropbox}

\declaretheoremstyle[
    headfont=\bfseries\sffamily\color{CorColor!70!black}, bodyfont=\normalfont,
    mdframed={
        linewidth=2pt,
        rightline=false, topline=false, bottomline=false,
        linecolor=CorColor, backgroundcolor=CorColor!5,
    }
]{thmcorbox}

\declaretheoremstyle[
    headfont=\bfseries\sffamily\color{ProbColor!70!black}, bodyfont=\normalfont,
    mdframed={
        linewidth=2pt,
        rightline=false, topline=false, bottomline=false,
        linecolor=ProbColor, backgroundcolor=ProbColor!5,
    }
]{thmprobbox}

\declaretheoremstyle[
    headfont=\bfseries\sffamily, bodyfont=\normalfont
]{thmexambox}



\declaretheorem[style=thmdefnbox, name=Definition, numberwithin=section]{defn}
\declaretheorem[style=thmthmbox, name=Theorem, numberwithin=section]{thm}
\declaretheorem[style=thmlemmabox, name=Lemma, numberwithin=section]{lemma}
\declaretheorem[style=thmpropbox, name=Proposition, numberwithin=section]{prop}
\declaretheorem[style=thmprobbox, name=Conjecture, numbered=no]{conj}
\declaretheorem[style=thmprobbox, name=Problem]{prob}
\declaretheorem[style=thmcorbox, name=Corollary, numberwithin=section]{cor}
\declaretheorem[style=thmexambox, name=Example, numberwithin=section]{eg}

% \newtheorem{thm}{Theorem}[section]
% \newtheorem{cor}[thm]{Corollary}
% \newtheorem{lmm}[thm]{Lemma}
% \newtheorem{prpn}[thm]{Proposition}
\theoremstyle{definition}
\newtheorem{rem}{Remark}[section]
\newtheorem{note}{Note}[section]

% \newtheorem{mainthm}{Theorem}
\declaretheorem[style=thmthmbox, name=Theorem]{mainthm}
\renewcommand{\themainthm}{\Alph{mainthm}}
\declaretheorem[style=thmlemmabox, name=Lemma, numbered=no]{lemmaSec}
\declaretheorem[style=thmthmbox, name=Theorem, numbered=no]{thmSec}


% \theoremstyle{definition}
% \newtheorem{defn}[thm]{Definition}
% \newtheorem{eg}[thm]{Example}

% \newtheorem*{conj}{Conjecture}
% \newtheorem*{thma}{Theorem A}
% \newtheorem*{thmb}{Theorem B}
% \newtheorem*{thmc}{Theorem C}
% \newtheorem*{thmd}{Theorem D}
\usepackage{import}
\usepackage{xifthen}
% \pdfminorversion=7
\usepackage{pdfpages}
\usepackage{transparent}
\newcommand{\incfig}[2][1]{%
    \def\svgwidth{#1\columnwidth}
    \import{./figures/}{#2.pdf_tex}
}

% https://tex.stackexchange.com/questions/1727/how-do-i-make-pages-which-were-intentionally-left-blank
% \makeatletter
%     \def\cleardoublepage{\clearpage%
%         \if@twoside
%             \ifodd\c@page\else
%                 \vspace*{\fill}
%                 \hfill
%                 \begin{center}
%                 This page intentionally left blank.
%                 \end{center}
%                 \vspace{\fill}
%                 \thispagestyle{empty}
%                 \newpage
%                 \if@twocolumn\hbox{}\newpage\fi
%             \fi
%         \fi
%     }
% \makeatother

% For blank page after new chapter
% \newcommand\blankpage{%
%     \null
%     \thispagestyle{empty}%
%     \addtocounter{page}{-1}%
%     \newpage}
% The following will automatic to the line breaking and hence avoid the overfull \hbox warning.
\sloppy 